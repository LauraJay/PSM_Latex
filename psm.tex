
\documentclass[times, 12pt,twocolumn]{article} 
\usepackage{latex8}
\usepackage{times}
\usepackage[ngerman]{babel}
\usepackage{todonotes}

\pagestyle{empty}


\begin{document}

\title{Neuronale Netze in der Videoproduktion}

\author{Laura Anger\\
Technische Hochschule K\"oln \\ Institut f\"ur Medien- und Phototechnik \\  laura.anger@th-koeln.de \\
\and
Vera Brockmeyer\\
Technische Hochschule K\"oln \\ Institut f\"ur Medien- und Phototechnik \\ vera-brockmeyer@smail.th-koeln.de \\
}

\maketitle
\thispagestyle{empty}

\large{\textbf{Zusammenfassung}}\\ \normalsize
\todo[inline, color=red]{Vera} 

\large{\textbf{Schl\"usselw\"orter}}\\ \normalsize
 Faltungsnetze, Videoproduktion
 \todo[inline, color=red]{Vera} 


\Section{Einleitung} \label{sec:Einleitung}
\todo[inline, color=red]{Vera} 
Die Videoproduktion konnte sich im Zuge der Digitalisierung im letzten Jahrzehnt enorm qualitativ verbessern. Viele Arbeitsprozesse, wie zum Beispiel die Verwaltung des 
\Section {Grundlagen} \label{sec:Grundlagen}
\todo[inline, color=red]{Laura} 
\SubSection{Videoproduktion} \label{sec:Videoproduktion}
\todo[inline, color=red]{Laura} 
\SubSection{Faltungsnetze} \label{sec:Faltungsnetze}
\todo[inline, color=red]{Laura} 

\Section{Konzeption} \label{sec:Konzeption}
\todo[inline, color=red]{Vera} 
\SubSection{Aktueller Stand} \label{sec:SOTAVorverarbeitung}
\todo[inline, color=red]{Vera} 

\Section{Produktion} \label{sec:Produktion}
\todo[inline, color=red]{Laura} 
\SubSection{Aktueller Stand} \label{sec:SOTAProduktion}
\todo[inline, color=red]{Laura} 

\Section{Postproduktion} \label{sec:Postroduktion}
\todo[inline, color=red]{Vera} 
\SubSection{Aktueller Stand} \label{sec:SOTAPostproduktion}
\todo[inline, color=red]{Vera} 

\Section{Zusammenfassung} \label{Zusammenfassung}
\todo[inline, color=red]{Vera} 





\bibliographystyle{latex8} 
\bibliography{latex8}

\end{document}

